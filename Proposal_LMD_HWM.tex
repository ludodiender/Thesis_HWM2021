\documentclass[twocolumn, 10pt, a4paper]{article}

% standard packages
\usepackage{titlesec, blindtext, color}                                  % standard packages
\usepackage[usenames,dvipsnames,svgnames,table]{xcolor} % extra colors 
\usepackage{graphicx}                                                       % figures
%\usepackage{natbib}                                                          % bibliography
\usepackage[english]{babel}                                              % correct hyphenation (afbreekstreepjes)
\usepackage{booktabs}                                                     % for midrule in tables
\usepackage{rotating}                                                        % for sdeways table
\usepackage{apacite}
\usepackage[modulo, switch]{lineno}

% PAGE MARGINS
\usepackage[top=2.5cm, bottom=2.5cm, left=2cm, right=2cm]{geometry}

% FONT
\usepackage[lf]{berenis}
\renewcommand*\familydefault{\sfdefault} 
\usepackage[T1]{fontenc}
% for other fonts, and how to install them, see the LaTeX Font Catalogue:
% http://www.tug.dk/FontCatalogue/

% LINE SPACE
\linespread{1.1}                          % more space between lines
\setlength{\parindent}{5mm}       % indenting first line paragraph
\linenumbers

% HYPHENATION (afbreekstreepjes)
% set words that are not abbreviated correctly  (expand list when necessary)
\hyphenation{catch-ment areas a-na-lyse}


%%%%%%%%%%%%%%%%%%%%%%%%%%%%%%%%%
%%%%%%%%%%%%%%%%%%%%%%%%%%%%%%%%%
% START
%%%%%%%%%%%%%%%%%%%%%%%%%%%%%%%%%
%%%%%%%%%%%%%%%%%%%%%%%%%%%%%%%%%

\begin{document}
	\chapter{\textbf{Preface}}
	\\
	Hi reader of this proposal!\\	
	When reading this pre-proposal for the thesis ring, I would highly appreciate feedback on the Problem Description, the Research Questions (as I'm not sure how to differentiate those from the objectives) and the Method section. For the Method section, I'm particularly interested in hearing how the explanation of Neural Networks is received (subsections 5.1 and 5.2). Thanks in advance!
	
	\title{\vspace{-1cm}\Huge{Improving rainfall rate predictions via Commercial Microwave Link signals in Sri Lanka using Deep Transfer Learning}}
	\author{\Large{Ludo Diender}}
	\date{\normalsize{MSc thesis proposal\\
			Hydrology and Quantitative Water Management Group,
			Wageningen University}}
	
	\maketitle
	
	\section{Problem description}
	
	In the past 25 years, Commercial Microwave Links (CML) have been recognized as a valuable opportunistic method to measure rainfall \cite{Leijnse2007} \cite{Ruf1996}. The links' signals get attenuated by rainfall by means of scattering and absorption. This attenuation is measured and stored by telecom companies for monitoring purposes and can be used to retrieve rainfall rates. Even more so in data scarce areas, where little precipitation is measured, CML's have proven to be an excellent addition to precipitation data \cite{Overeem2021,Doumounia2014,Diba2021}. Their density in populated areas makes them especially useful for urban hydrology where high spatial and temporal resolution of precipitation data is required \cite{Overeem2011}. In general, having ample and correct precipitation data is important for flood warnings, agriculture, river systems, shipping routes and many more \cite{Chwala2019}. 
	
	The first studies on the use of CML signals to retrieve rainfall rates were done by using a specific Power-Law (PL) to relate the attenuation of the signal and the rainfall rate \cite{Overeem2011,Leijnse2007}. This method, which includes a wet-dry classification, baseline estimation, wet antenna attenuation estimation and finally a rainfall rate retrieval, has yielded good results in multiple studies \cite{deVos2019,Graf2020,Fencl2017}. Recently methods have shifted away from this PL algorithm and used a more data-driven approach in the form of neural networks of different architectures. Studies have been performed in Sweden, Israel \cite{Habi2019}, Germany \cite{Polz2020}, South Korea and Ethiopia \cite{Diba2021} on the use of such data-driven networks in relating CML signals to rainfall rates. Previous studies have shown that such data-driven models can be more accurate, less time-demanding and more robust in estimating rainfall rates compared to the PL method \cite{Polz2020,Pudashine2020}. Neural networks are not a novelty in predicting rainfall \cite{French1992}, but the application to CML data is recently been rising in popularity.
		
	One of the disadvantages of using data-driven methods like neural networks, apart from the black-box characteristics inherent to the method, is the dependency on a large training data set. In areas with less or little available training data, transfer learning provides the opportunity to adapt an already existing model with a certain structure to do a sligthly different task \cite{TanYear}. The technique exploits the availability of data in the source domain and is able to transfer the knowledge to the target domain. It does so by relaxing the underlying assumption that training and test data for a Machine Learning model should be independent and identically distributed \cite{Weiss2016}. Although used quite often in different applications \cite{Zhuang2021}, the concept of transfer learning has not yet been used to improve the precipitation estimation using CML's in data-scarce areas. 
		
	Recent research focused on the use of CML data to measure rainfall in tropical regions, more specifically Sri Lanka \cite{Overeem2021} and Brazil \cite{RiosGaona2017a}. The relatively small amount of reference rain gauges in Sri Lanka especially made this research more challenging compared to well-equipped countries like the Netherlands \cite{Overeem2013}. Both of the two studies mentioned above (Sri Lanka and the Netherlands) are based on the algebraic PL method. There have not been any efforts yet to analyze the potential of CML's for rainfall retrieval using data-driven methods for neither the Netherlands nor Sri Lanka. 

	
	\section{Research objectives}
	
	The objectives of this research are to:
	\begin{itemize}
		\item train, test and validate a neural network on CML data in the Netherlands to obtain precipitation rates
		\item apply the concept of transfer learning to make the model able to obtain precipitation rates for Sri Lanka
		\item create 2D interpolated rainfall maps using the neural network architecture
		\item compare the data-driven approach to previous approaches.
	\end{itemize} 
	With these four objectives, this research will give an indication of the potential of transfer learning and data-driven methods for areas with little reference data like Sri Lanka. 
	 
	
	\section{Research questions}
	The research questions all relate to the objectives mentioned above. 
	\begin{itemize}
		\item How does a neural network perform on Dutch data?
		\item How does transfer learning improve the use of CML's in Sri Lanka as precipitation measurements?
		\item What is the potential of the use of NN for 2D interpolation of rainfall maps
		\item Does a neural network outperform previous methods like the power-law algorithm in retrieving rainfall rates from CML's?
	\end{itemize}
	
	
	\section{Field site and data}
	\textbf{I will write this section later, skip this for now when reading!}
	Making use of Aart's dataset of the Netherlands and the provided data for Sri Lanka. Combination makes the research come together. Explain the type of signal data that I got (quantity, quality, metadata)
	
	\section{Methods}
	\subsection{Neural networks}
	To conduct this research, I will make use of a neural network. Neural networks are part of deep learning. An input signal is transferred through different layers of the model. Every layer consists of different neurons (or nodes) that are able to extract features from the input signal. Every node consists of a weighted sum of the complete input signal. The weights that are given to the different values in this input signal, together with a bias that is added to this weighted sum, make up the parameter set of the neuron. 
	The subsequent layers extract combinations out of the previous layer, until finally the network ends up with a prediction with a certain probability. By applying a loss function, the network can learn how to improve, by changing the different parameters in all neurons. The to-be-extracted features are unknown beforehand to the researcher. The network itself learns which features are interesting, separable and are helpful in classifying the specific signal. This is the cornerstone of machine learning/deep learning. 
	
	In this research I will use a Long Short Term Memory (LSTM) architecture for my neural network, which is part of a Recurrent Neural Networks (RNN). RNN's a characterized by the recurrent use of the same bit of network, to keep on improving the prediction. The network has one or two layers, depending on the complexity of the features and the signal. The signal is then processed multiple times by the same two layers. RNN's are designed for sequential data like the time signals that are used in this research. A disadvantage of RNN's is the risk of vanishing or exploding gradients, due to the lack of memory in the network. LSTM's resolve this issue by adding a memory to the model by using different gates to combine old and new data on every recurrent step of the network. The use of LSTM's to create a network for CML data has been demonstrated before \cite{Habi2019, Diba2021, Pudashine2020}. The networks used in those studies will serve as a source of inspiration for this research. Other types of neural networks have been proposed as well \cite{Polz2020}, but as RNN's are designed to work with sequential data, those are preferred in this research.
	
	\subsection{Training, testing and transferring}
	The  Dutch dataset as described in section 4 will be split up in two sets. One is dedicated to training the model and the other one will be used for testing. The training data is passed through the LSTM, the loss is calculated for each training sample and the network 'learns' (is updated) by using backward propagation. Backward propagation is a machine learning technique where the final layers are updated first and the updates move backwards through the model. 
	
	After training for a certain number of epochs (evolutions, number yet to be determined) the model is tested using the test data. The performance on the testing data is the most important and determines the performance of the model overall. 
	
	Once the LSTM performs well, transfer learning will be applied, to overhaul the information and knowledge gained by the model to quickly adapt to Sri Lankan data. By removing the outer layer of the model, the feature extraction part remains and can be used for a different dataset with different characteristics as well. The transferred model will, similarly to the first model, be trained on a subset of the Sri Lankan dataset. Afterwards, it will be tested on the remaining data to evaluate the potential of the transferred model. 
	
	\subsection{2-D interpolated rainfall maps}
	\textbf{This section still needs some extra attention, please skip this for now}
	CNN's have been proposed to deal with spatial patterns in rainfall data \cite{Sadeghi2019}. Could use radar images to learn the patterns of the Netherlands and subsequently use the CML data to recognize these patterns and create a rainfall map for the whole country using these CNN's. Might not result in anything different than the maps that are already made using the IDW or Nearest Neighbour techniques. 
	
	\subsection{Comparison to PL algorithm}
	\textbf{This section still needs some extra attention, please skip this for now}	
	
	\section{Timetable}
	
	Adapt the Table below to make it specific for your project (or make your own). Set deadlines for the products. Be as specific as possible: mention when you will collect which data /do which model runs / write which parts of the report. It often helps to link activities and products to your sub-questions. A specific planning can help later on to see if you are on schedule or that you should e.g. shorten a certain data-processing step or stop calibrating your model, so you have enough time to do the analyses and answer your research questions. It often helps to link the tasks to the methodology (and therefore to the research questions). Specify special conditions: are you planning to take courses, vacation, etc. 
	
	
	% table with schedule
	\begin{table*}[t]
		\caption{Schedule of the project.}
		\label{tab:schedule}
		% \includegraphics[width=2.1\columnwidth, trim=5mm 80mm 5mm 18mm, clip=true]{figs/thesis_planning.pdf}
	\end{table*}
	% If not the whole table is shown: adjust the numbers behind "trim". 
	% Those are the cropped sides in the order left - bottom - right - top.
	
	
	
	
	%%%%%%%%%%%%%%%%%%%%%%%%%%%%%%%%%
	% BIBLIOGRAPHY
	%%%%%%%%%%%%%%%%%%%%%%%%%%%%%%%%%
	
	% \renewcommand{\bibname}{Bibliography} 
	\bibliographystyle{apacite}
	\bibliography{Proposal_Ludo}
	
	% to add items to the bibliography:
	% option 1. open "references_thesis.bib" in JabRef (free downloadable), enter more papers
	% option 2. open "references_thesis.bib" in NotePad, go to Google Scholar, find paper, click "cite", click "import into BiBTeX", copy text into NotePad
	
	
	
	%%%%%%%%%%%%%%%%%%%%%%%%%%%%%%%%%
	% END
	%%%%%%%%%%%%%%%%%%%%%%%%%%%%%%%%%
	
	
\end{document}

